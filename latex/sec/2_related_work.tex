\section{Related Work}
\subsection{Brain-Inspired Perception}
\hspace{1pc}The human visual cortex possesses remarkable environmental perception capabilities, serving as a crucial component of the central nervous system and responsible for transforming visual signals into comprehensible information \cite{9134376}. Traditional visual encoding models are limited by the use of either handcrafted or deep learning features merely, despite their individual advantages. To integrate the two, Cui et al. \cite{8574054} proposed the GaborNet-VE model, which combines Gabor features with deep learning to form an efficient visual encoding framework. However, this model requires significant parameter adjustments across different tasks, and the interpretability of its deep learning component remains inadequate. In contrast, our proposed BID  model demonstrates outstanding performance in revealing decision-making mechanisms and exhibiting strong generalization capabilities.
\subsection{Brain-Inspired Decision}
\hspace{1pc}In practical environments, agents often need to handle continuous state spaces, while traditional reinforcement learning models, such as Q-learning and Actor-Critic algorithms, are more suitable for discrete states. To address this issue, Zhao et al. \cite{zhao2018brain} proposed the prefrontal cortex-basal ganglia (PFC-BG) algorithm, which subdivides continuous states and introduces continuous reward functions to capture temporal reward information. However, the design of continuous reward functions remains a challenge. In non-Gaussian and nonlinear environments (NGNLE), traditional methods struggle to cope with their nonlinear and non-Gaussian characteristics. Inspired by human brain decision-making, Naghshvarianjahromi et al. \cite{8932505} designed an autonomous computation layer based on a cognitive dynamic system (CDS), providing agents in NGNLE with stronger decision-making capabilities. Although these methods attempt to mimic the decision-making process of the human brain, structural differences lead to a lack of transparency in their processes. Therefore, it is crucial to construct network models that align with the anatomical structure of human brain.