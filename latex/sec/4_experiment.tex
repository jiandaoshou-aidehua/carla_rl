\section{Experiment}
\begin{figure}[t]
	\centering
	%\fbox{\rule{0pt}{2in} \rule{0.9\linewidth}{0pt}}
	\includegraphics[width=0.8\linewidth]{experiment.png}
	
	\caption{Assessing the Similarity Between Network Activation and Brain Activation.}
	\label{fig:fig3}
\end{figure}
\label{sec:experiment}
\hspace{1pc}\textbf{\textsf{Implementation Details.}} During the experiment, we randomly selected start and target positions, and computed the route using A* algorithm. Subsequently, we executed $\pi_{\theta_{k}}$ in the CARLA \cite{Dosovitskiy17} environment to gather trajectory data and synchronized brain activation record. We employed a specific reward scheme and imposed additional penalties for large steering changes to prevent oscillatory maneuvers. After training the network, we compared the similarity between the network activations and those observed in the human brain. As depicted in Fig. \ref{fig:fig3}, we contrasted the similarity between the test participant and the brain-inspired network in processing the same event.

\textbf{\textsf{Performance of BID.}} The performance of the BID agent is constrained by the performance of the expert it is imitating. When the expert performs poorly, the agent that mimics the expert will also exhibit poor performance. The BID model is designed in strict accordance with the human brain's visual information processing process, both structurally and functionally. When optimized, the model's output results are very close to those of the human brain in terms of evaluating the similarity in activation. 